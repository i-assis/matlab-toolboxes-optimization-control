%% =============================
%%      IMPORTANTE
%% ESTE ARQUIVO DEVE ESTAR SALVO COMO
%%      UTF - 8
%% =============================

% ----------------------------------------------------------
% Este capítulo é parte integrante do arquivo mestre
% Relatorio_TCC_Mestrado_Base_VERSÃO_SUBVERSÃO_FHZ
% ----------------------------------------------------------


% ----------------------------------------------------------
\chapter{Objetivos}
\label{cap_objetivos}
% ----------------------------------------------------------

Estes são os objetivos.

\begin{equation} \label{obj_eq_teste}
	Teste-para-referencias-em-arquivos-distintos
\end{equation}

Este arquivo serve de modelo e tutorial para uso em dissertações de mestrado e teses de doutorado (qualificação e defesa) usados na UFABC.

A \autoref{tab_siglas_cursos_ufabc} apresenta as siglas e os respectivos cursos para serem inseridos na opção da classe
{\ufabcFHZ}
na linha de comando
\verb|\department{ }|.

A classe
{\ufabcFHZ}
também possui a opção de alterar o idioma automaticamente adicionando a opção
``\textbf{EN}''
ao comando
``\verb|\documentclass[ ]|''
conforme consta nos comentários do arquivo base reproduzidos a seguir:

\begin{verbatim}
	\documentclass[msc]{ufabcFHZh}
	%\documentclass[msc, EN]{ufabcFHZh}
	% ---- Opções da classe:
	%%% msc 	- Mestrado  - Dissertação
	%%% dscexam	- Doutorado - Exame de Qualificação
	%%% dsc 	- Doutorado - Tese
	%%% -- // -- // -- Seleção de idiomas, PT é padrão em caso de vazio/omissão
	%%% É necessário compilar duas vezes seguidas para 
	atualizar o arquivo após alterar o idioma
	%%% (Vazio) - Português
	%%% EN - English
	% ----
\end{verbatim}

% ----------------- Tabela Departamentos
\begin{table}[H]
	\centering
	\caption{Tabelas de abreviações para inserir nome de curso neste modelo}
	\label{tab_siglas_cursos_ufabc}
	\small
	\begin{tabular}{r|l}
		\multicolumn{2}{c}{Opções de departmento}                                                                        \\ \hline
		\multicolumn{2}{c}{Edição em ufabcFHZ\#.cls}                                                                    \\ \hline
		\{PEB\}                                              & \{Engenharia Biomédica\}                                             \\
		\{PEC\}                                              & \{Engenharia Civil\}                                                 \\
		\{PEE\}                                              & \{Engenharia Elétrica\}                                              \\
		\{PEM\}                                              & \{Engenharia Mecânica\}                                              \\
		\{PEMM\}                                             & \{Engenharia Metalúrgica e de Materiais\}                            \\
		\{PEN\}                                              & \{Engenharia Nuclear\}                                               \\
		\{PENO\}                                             & \{Engenharia Oceânica\}                                              \\
		\{PPE\}                                              & \{Planejamento Energético\}                                          \\
		\{PEP\}                                              & \{Engenharia de Produção\}                                           \\
		\{PEQ\}                                              & \{Engenharia Química\}                                               \\
		\{PESC\}                                             & \{Engenharia de Sistemas e Computação\}                              \\
		\{PET\}                                              & \{Engenharia de Transportes\}                                        \\\hline
		\multicolumn{1}{c}{Adicionado em 18.11.2015} & \multicolumn{1}{c}{fonte: \url{http://propg.ufabc.edu.br/cursos/}} \\\hline
		\{BIOS\}                                             & \{Biossistemas\}                                                     \\
		\{BIOT\}                                             & \{Biotecnociência\}                                                  \\
		\{CCM\}                                              & \{Ciência da Computação\}                                            \\
		\{CTA\}                                              & \{Ciência e Tecnologia Ambiental\}                                   \\
		\{CTQ\}                                              & \{Ciência e Tecnologia/Química\}                                     \\
		\{CHS\}                                              & \{Ciências Humanas e Sociais\}                                       \\
		\{EEN\}                                              & \{Energia\}                                                          \\
		\{MEB\}                                              & \{Engenharia Biomédica\}                                             \\
		\{MEI\}                                              & \{Engenharia da Informação\}                                         \\
		\{EGI\}                                              & \{Engenharia e Gestão da Inovação\}                                  \\
		\{EHFCM\}                                            & \{Ensino, História e Filosofia das Ciências e Matemática\}           \\
		\{EDI\}                                              & \{Evolução e Diversidade \}                                          \\
		\{FIL\}                                              & \{Filosofia\}                                                        \\
		\{FIS\}                                              & \{Física\}                                                           \\
		\{MAT\}                                              & \{Matemática\}                                                       \\
		\{NMA\}                                              & \{Nanociências e Materiais Avançados\}                               \\
		\{NCG\}                                              & \{Neurociência e Cognição\}                                          \\
		\{PGT\}                                              & \{Planejamento e Gestão do Território\}                              \\
		\{PPU\}                                              & \{Políticas Públicas\}                                               \\ %\hline
		%\multicolumn{2}{c}{Mestrado Profissional}                                                                                   \\ \hline
		\{PROFMAT\}                                          & \{Mestrado Profissional em Matemática em Rede Nacional\}             \\
		\{MNPEF\}                                            & \{Mestrado Nacional Profissional em Ensino de Física\}               \\ %\hline
		%\multicolumn{2}{c}{Doutorado Acadêmico Industrial}                                                                          \\ \hline
		\{DAI\}                                              & \{Doutorado Acadêmico Industrial\}                                             
	\end{tabular}
\end{table}
% -----------------

% ----------------------------------------------------------
% Fim Arquivo